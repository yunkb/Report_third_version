\chapter*{Abstract}
State estimation algorithms are widely used on transmission grids for estimating the most-likely state of the system. In this thesis, four state estimation algorithms, including one newly developed estimator which combines Extended Kalman Filter (EKF) and Support Vector Regression (SVR) together, are implemented and tested on a balanced low voltage radial distribution grid for different scenarios characterized by different levels and types of measurements. For each scenario, 100 cases are studied whose location of the smart meters are selected randomly. The power injections at 206 buses comes from actual Photovoltaic (PV) generators and small household loads collected from the City of Basel, and pseudo-measurements are generated to achieve observability of the system in case of partial penetration of sensors. After power flow calculation and state estimation, the performance of four state estimation algorithms on the distribution grid are compared, notably considering the influence of the location of the smart meters on the accuracy of estimators. And a new methodology to install smart meters at the optimal place with high accuracy for state estimation algorithms is proposed. Based on the proposed optimal smart meter location, four estimators are compared again with the previous results from random smart meter location. Finally, the accuracy of each estimator regarding voltage magnitude error, active power flow error, and reactive power flow error is analyzed and compared. Despite the fact that multiple algorithms specific for distribution grid state estimation are suggested in the literature, the classical Weighted Least Square (WLS) algorithm still provides the best state estimation.


\newpage